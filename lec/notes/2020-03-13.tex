\documentclass[12pt, leqno]{article}
\usepackage{fancyhdr}
\usepackage[sort&compress]{natbib}
\usepackage[letterpaper=true,colorlinks=true,linkcolor=black]{hyperref}

\usepackage{amsfonts}
\usepackage{amsmath}
\usepackage{amssymb}
\usepackage{color}
\usepackage{tikz}
\usepackage{pgfplots}
\usepackage{listings}
%\usepackage{courier}
%\usepackage[utf8]{inputenc}
%\usepackage[russian]{babel}

\lstdefinelanguage{Julia}%
  {morekeywords={abstract,break,case,catch,const,continue,do,else,elseif,%
      end,export,false,for,function,immutable,import,importall,if,in,%
      macro,module,otherwise,quote,return,switch,true,try,type,typealias,%
      using,while},%
   sensitive=true,%
   alsoother={$},%
   morecomment=[l]\#,%
   morecomment=[n]{\#=}{=\#},%
   morestring=[s]{"}{"},%
   morestring=[m]{'}{'},%
}[keywords,comments,strings]%

\lstset{
  numbers=left,
  basicstyle=\ttfamily\footnotesize,
  numberstyle=\tiny\color{gray},
  stepnumber=1,
  numbersep=10pt,
}

\newcommand{\iu}{\ensuremath{\mathrm{i}}}
\newcommand{\bbR}{\mathbb{R}}
\newcommand{\bbC}{\mathbb{C}}
\newcommand{\calV}{\mathcal{V}}
\newcommand{\calE}{\mathcal{E}}
\newcommand{\calG}{\mathcal{G}}
\newcommand{\calW}{\mathcal{W}}
\newcommand{\calP}{\mathcal{P}}
\newcommand{\macheps}{\epsilon_{\mathrm{mach}}}
\newcommand{\matlab}{\textsc{Matlab}}
\newcommand{\uQ}{\underline{Q}}
\newcommand{\uR}{\underline{R}}

\newcommand{\ddiag}{\operatorname{diag}}
\newcommand{\fl}{\operatorname{fl}}
\newcommand{\nnz}{\operatorname{nnz}}
\newcommand{\tr}{\operatorname{tr}}
\renewcommand{\vec}{\operatorname{vec}}

\newcommand{\vertiii}[1]{{\left\vert\kern-0.25ex\left\vert\kern-0.25ex\left\vert #1
    \right\vert\kern-0.25ex\right\vert\kern-0.25ex\right\vert}}
\newcommand{\ip}[2]{\langle #1, #2 \rangle}
\newcommand{\ipx}[2]{\left\langle #1, #2 \right\rangle}
\newcommand{\order}[1]{O( #1 )}

\newcommand{\kron}{\otimes}


\newcommand{\hdr}[2]{
  \pagestyle{fancy}
  \lhead{Bindel, Fall 2020}
  \rhead{Numerical Analysis}
  \fancyfoot{}
  \begin{center}
    {\large{\bf HW for #1}} \\ (due: #2)
  \end{center}
  \lstset{language=Julia,columns=flexible}
}


\begin{document}
\hdr{2020-03-13}

\section{Nonlinear equations and optimization}

If $f : \bbR^n \rightarrow \bbR^n$, then solving the system $f(x) = 0$
is equivalent to minimizing $\|f(x)\|^2$.  Similarly, if $g : \bbR^n
\rightarrow \bbR$ is continuously differentiable, then any local
minimizer $x_*$ satisfies the nonlinear equations $\nabla g(x_*) = 0$.
There is thus a close connection between nonlinear equation solving
on the one hand and optimization on the other, and methods used for
one problem can serve as the basis for methods for the other.

As with nonlinear equations, the one-dimensional case is the simplest,
and may be the right place to start our discussion.  As
with the solution of nonlinear equations, our main strategy for
dealing with multi-variable optimization problems will be to find a
promising search direction and then solve (approximately) a
one-dimensional line search problem.

\section{Minimization via 1D Newton}

Suppose $g : \bbR \rightarrow \bbR$ has at least two continuous derivatives.
If we can compute $g'$ and $g''$, then one of the simplest ways to find a
local minimum is to use Newton iteration to find a stationary point:
\[
  x_{k+1} = x_k - \frac{g'(x_k)}{g''(x_k)}.
\]
Geometrically, this is equivalent to finding the maximum
(or minimum) of a second-order Taylor expansion about $x_{k}$;
that is, $x_{k+1}$ is chosen to minimize (or maximize)
\[
  \hat{g}(x_{k+1}) = g(x_k) + g'(x_k)(x_{k+1}-x_k) + \frac{1}{2} g''(x_k) (x_{k+1}-x_k)^2.
\]
The details are left as an exercise.

There are two gotchas in using Newton iteration in this way.  We have
already run into the first issue: Newton's method is only locally
convergent.  We can take care of that problem by combining Newton with
bisection, or by scaling down the length of the Newton step.  But there
is another issue, too: saddle points and local maxima are also stationary
points!

There is a simple precaution we can take to avoid converging to a maximum:
insist that $g(x_{k+1}) < g(x_k)$.  If $x_{k+1} = x_k - \alpha_k u$ for some
$\alpha_k > 0$, then
\[
  g(x_{k+1}) - g(x_k) = -\alpha_k g'(x_k) u + O(\alpha_k^2).
\]
So if $g'(x_k) u > 0$, then $-u$ is a {\em descent direction}, and thus
$g(x_{k+1}) < g(x_k)$ provided $\alpha_k$ is small enough.  Note that if
$x_k$ is not a stationary point, then $-u = -g'(x_k)/g''(x_k)$ is
a descent direction iff $g'(x_k) u = g'(x_k)^2 / g''(x_k) > 0$.  That
is, we will only head in the direction of a minimum if $g''(x_k)$ is
positive.  Of course, $g''$ will be positive and the Newton step will
take us in the right direction if we are close enough to a strong
local minimum.

\section{Approximate bisection and golden sections}

Assuming that we can compute first derivatives, minimizing in 1D reduces
to solving a nonlinear equation, possibly with some guards to prevent the
solver from wandering toward a solution that does not correspond to a
minimum.  We can solve the nonlinear equation using Newton iteration,
secant iteration, bisection, or any combination thereof, depending how
sanguine we are about computing second derivatives and how much we are
concerned with global convergence.  But what if we don't even want to
compute first derivatives?

To make our life easier, let's suppose we know that $g$ is twice
continuously differentiable and that it has a unique minimum at
some $x_* \in [a,b]$.  We know that $g'(x) < 0$ for $a \leq x < x_*$ and
$g'(x) > 0$ for $x_* < x \leq b$; but how can we get a handle on $g'$ without
evaluating it?  The answer lies in the mean value theorem.  Suppose we
evaluate $g(a)$, $g(b)$, and $g(x)$ for some $x \in (a,b)$.
What can happen?
\begin{enumerate}
\item
  If $g(a)$ is smallest ($g(a) < g(x) \leq g(b)$),
  then by the mean value theorem,
  $g'$ must be positive somewhere in $(a,x)$.  Therefore, $x_* < x$.
\item
  If $g(b)$ is smallest, $x_* > x$.
\item
  If $g(x)$ is smallest, we only know $x_* \in [a,b]$.
\end{enumerate}
Cases 1 and 2 are terrific, since they mean that we can improve
our bounds on the location of $x_*$.  But in case 3, we have no
improvement.  Still, this is promising.  What could we get from
evaluating $g$ at {\em four} distinct points $a < x_1 < x_2 < b$?
There are really two cases, both of which give us progress.
\begin{enumerate}
\item
  If $g(x_1) < g(x_2)$ (i.e.~$g(a)$ or $g(x_1)$ is smallest) then $x_* \in [a, x_2]$.
\item
  If $g(x_1) > g(x_2)$ (i.e.~$g(b)$ or $g(x_2)$ is smallest) then $x_* \in [x_1,b]$.
\end{enumerate}
We could also conceivably have $g(x_1) = g(x_2)$, in which case the minimum
must occur somewhere in $(x_1,x_2)$.

There are now a couple options.  We could choose $x_1$ and $x_2$ to be
very close to each other, thereby nearly bisecting the interval in all
four cases.  This is essentially equivalent to performing a step of
bisection to find a root of $g'$, where $g'$ at the midpoint is
estimated by a finite difference approximation.  With this method, we
require two function evaluations to bisect the interval, which means
we narrow the interval by $1/\sqrt{2} \approx 71\%$ per evaluation.

We can do a little better with a {\em golden section search}, which
uses $x_2 = a+(b-a)/\phi$ and $x_1 = b + (a-b)/\phi$, where $\phi =
(1+\sqrt{5})/2$ (the {\em golden ratio}).  We then narrow to the
interval $[a,x_2]$ or to the interval $[x_1,b]$.  This only narrows
the interval by a factor of $\phi^{-1}$ (or about 61\%) at each step.
But in the narrower interval, we get one of the two interior function
values ``for free'' from the previous step, since $x_1 =
x_2+(a-x_2)/\phi$ and $x_2 = x_1+(b-x_1)/\phi$.  Thus, each step only
costs one function evaluation.

\section{Successive parabolic interpolation}

Bisection and golden section searches are only linearly convergent.
Of course, these methods only use coarse information about the
relative sizes of function values at the sample points.
In the case of root-finding, we were able to get a superlinearly
convergent algorithm, the secant iteration, by replacing the linear
approximation used in Newton's method with a linear interpolant.
We can do something similar in the case of optimization by interpolating
$g$ with a {\em quadratic} passing through three points, and then finding
a new guess based on the minimum of that quadratic.  This {\em method
of successive parabolic interpolation} does converge locally superlinearly.
But even when $g$ is unimodular, successive parabolic interpolation must
generally be coupled with something slower but more robust
(like golden section search) in order to guarantee
good convergence.

\section*{Problems to ponder}

\begin{enumerate}
\item
  Suppose I know $f(0)$, $f(1)$, and a bound $|f''| < M$ on $[0,1]$.
  Under what conditions could $f$ possibly have a local minimum in
  $[0,1]$?
\item
  Suppose $f(x)$ is approximated on $[0,1]$ by a polynomial $p(x) =
  c_0 + c_1 x + \ldots + c_d x^d$, and we know that $|f(x)-p(x)| <
  \delta$ on the interval.  Using MATLAB's {\tt roots} function, how
  could we find tight subintervals of $[0,1]$ in which the global
  minimum of $f(x)$ might lie?
\end{enumerate}

\end{document}
